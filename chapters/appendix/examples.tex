\documentclass[../../main.tex]{subfiles}

\begin{document}
\section{Tables}
Inserting a table with values, like in Table~\ref{tab:table}.

\begin{table}[htb!]
    \centering
    \begin{tabular}{>{\bfseries}lcc}
        \toprule
        Størrelse   & $O(n^2)$  & $O(n!)$ \\
        \midrule
        2           & 4         & 2 \\
        3           & 9         & 6 \\
        4           & 16        & 24 \\
        5           & 25        & 120 \\
        10          & 100       & 3628800 \\
        15          & 225       & $\approx 1.30767 \times 10^{12}$ \\
        \bottomrule
    \end{tabular}
    \caption{Comparing the growth of $O(n^2)$ and $O(n!)$, as $n$ increases.}
    \label{tab:table}
\end{table}


\section{Listings}
We can use the listing float for code, as seen in Listing~\ref{lst:listing}.

\begin{listing}[htb!]
    \begin{minted}[highlightlines={1,3}]{rust}
        fn main() -> i32 {
            !println("Hello World!");
            return 0;
        }
    \end{minted}
    \caption{Styling of Rust code, defined in the preamble.}
    \label{lst:listing}
\end{listing}

\clearpage

\section{Subfigures}
It's possible to ``nest'' figures

\begin{figure}[htb!]
%%%%%%%%%%%%%%%%%%%%%% 1 %%%%%%%%%%%%%%%%%%%%%%%%%%%%%%%%%%%
\begin{subfigure}[b]{0.45\textwidth}
    \begin{tikzpicture}[>={Latex[length=3mm,width=4mm]}]
        % place nodes
        \node[draw,circle] at (0,0) (home) {S};
        \node[draw,circle] at (-1,2)  (one) {1};
        \node[draw,circle] at (3,0.5)  (two) {2};
        \node[draw,circle] at (4.5, 4.5)  (three) {3};
        % draw edges
        \path[-]
        (home) edge[dashed] node[left] {750 m} (one)
        (home) edge[dashed] node[above] {900 m} (two)
        (one) edge[dashed] node[above=0.15] {1250 m} (two)
        (one) edge[dashed] node[above=0.12] {1800 m} (three)
        (two) edge[dashed] node[right] {1250 m} (three)
        (three) edge[dashed] node[right] {1900 m} (home);
    \end{tikzpicture}
    \caption{The distance between each node in meters. 1, 2 and 3 are nodes we have to visit exactly once and $S$ is both the start and end node.}
    \label{fig:nn-vs-optimal-1}
\end{subfigure}
\hfill
%%%%%%%%%%%%%%%%%%%%%% 2 %%%%%%%%%%%%%%%%%%%%%%%%%%%%%%%%%%%
\begin{subfigure}[b]{0.45\textwidth}
    \begin{tikzpicture}[>={Latex[length=3mm,width=4mm]}]
        % place nodes
        \node[draw,circle] at (0,0) (home) {S};
        \node[draw,circle] at (-1,2)  (one) {1};
        \node[draw,circle] at (3,0.5)  (two) {2};
        \node[draw,circle] at (4.5, 4.5)  (three) {3};
        % draw edges
        \path[<-]
        (home) edge node[above] {900 m} (two)
        (two) edge node[right] {1250 m} (three)
        (three) edge node[above = 0.15] {1800 m} (one)
        (one) edge node[right] {750 m} (home);
    \end{tikzpicture}
    \caption{The most optimal route to take, in order to achieve the shortest possible distance. Total distance of 4700m.}
    \label{fig:nn-vs-optimal-2}
\end{subfigure}
\vskip\baselineskip
%%%%%%%%%%%%%%%%%%%%%% 3 %%%%%%%%%%%%%%%%%%%%%%%%%%%%%%%%%%%
\begin{subfigure}[b]{0.45\textwidth}
    \begin{tikzpicture}[>={Latex[length=3mm,width=4mm]}]
        \node[draw,circle] at (0,0) (home) {S};
        \node[draw,circle] at (-1,2)  (one) {1};
        \node[draw,circle] at (3,0.5)  (two) {2};
        \node[draw,circle] at (4.5, 4.5)  (three) {3};
        \path[->]
        (home) edge node[right] {750 m} (one)
        (one) edge[dashed, color=red] node[above=0.15] {1250 m} (two)
        (one) edge[dashed] node[above=0.15] {1800 m} (three);
    \end{tikzpicture}
    \caption{\textsc{Nearest Neighbor} will not find the shortest route.
    After choosing the edge $c_{S,1}$, the edge $c_{1,2}$ will be the one with the lowest weight and therefore be chosen.}
    \label{fig:nn-vs-optimal-3}
\end{subfigure}
\hfill
%%%%%%%%%%%%%%%%%%%%%% 4 %%%%%%%%%%%%%%%%%%%%%%%%%%%%%%%%%%%
\begin{subfigure}[b]{0.45\textwidth}
    \begin{tikzpicture}[>={Latex[length=3mm,width=4mm]}]
        \node[draw,circle] at (0,0) (home) {S};
        \node[draw,circle] at (-1,2)  (one) {1};
        \node[draw,circle] at (3,0.5)  (two) {2};
        \node[draw,circle] at (4.5, 4.5)  (three) {3};
        \path[->]
        (home) edge node[right] {750 m} (one)
        (one) edge node[above=0.15] {1250 m} (two)
        (two) edge node[right] {1250 m} (three)
        (three) edge node[right] {1900 m} (home);
    \end{tikzpicture}
    \caption{The full route taken, when using \textsc{Nearest Neighbor}, as a result of~\ref{fig:nn-vs-optimal-3}. Total distance of 5150m.}
    \label{fig:nn-vs-optimal-4}
\end{subfigure}
\caption{An example of \textsc{Nearest Neighbor} not always finding the shortest route.
The dashed lines show the distance between each node, the dashed arrows show the possible edges we can take, the arrows show the route we have taken and the red node is the current location.
In this case, the shortest route has a total distance of 4700m, but with \textsc{Nearest Neighbor} the distance is 5150m.}
\label{fig:nn-vs-optimal}
\end{figure}


\section{Examples}
If you wish to showcase an example, it can be done like this:

\begin{exa}
    I am an example. To show that $x^2 + y^2 = z^2$.
\end{exa}


\section{Algorithms}
An alogrithm environment, based on the \texttt{clrscode3e} package, which is used in the book \citetitle{alg-book} by \citeauthor{alg-book} \cite{alg-book}.

\begin{algorithm}
    \begin{codebox}
        \Procname{$\proc{Insertion-Sort}(A)$}
        \li \For $j \gets 2$ \To $\attrib{A}{length}$
        \li \Do
        $\id{key} \gets A[j]$
        \li \Comment Insert $A[j]$ into the sorted sequence
        $A[1 \twodots j-1]$.
        \li $i \gets j-1$
        \li \While $i > 0$ and $A[i] > \id{key}$
        \li \Do
        $A[i+1] \gets A[i]$
        \li $i \gets i-1$
        \End
        \li $A[i+1] \gets \id{key}$
        \End
    \end{codebox}
    \caption{Example of pseudocode (codebox) in an algorithm float.}
    \label{alg:algorithm}
\end{algorithm}
\end{document}
